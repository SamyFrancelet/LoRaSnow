\section{Analyse de la clientèle}
\subsection{Administration publique}
Les administrations publiques verront en ce projet la possibilité de superviser
un territoire parfois compliqué (par exemple, les fonds de vallée,
routes de montagnes, etc...) et d'efficacement déneiger ou saler
afin que la chaussée soit prête dès que possible à accueillir des automobilistes.
Par exemple, la commune d'Ayent, étant très vaste et possédant des zones
où l'accès est compliqué, fait face au problème de la répartition inhomogène
des chutes de neige. Durant la nuit, il est donc contraignant d'envoyer
quelqu'un contrôler chaque zone. Des allers-retours inutiles peuvent être
évités grâce à un réseau de capteurs.
De plus, malgré la mise en place de personnel de piquet, il est possible
d'être surpris par des chutes de neige non annoncées par la météo.

\subsection{Entreprise privée}
De nombreuses entreprises fournissent des services de déneigement pour particulier.
Installer des capteurs chez les clients (par exemple devant chez un boulanger),
permettrai de mieux planifier le déneigement, et ainsi d'améliorer la qualité
du service offert aux particuliers.
Notre solution permet également de minimiser les temps de sortie des véhicules,
et ainsi réaliser des économies.
Intégrer à tous leurs clients, LoRaSnow donnera une vue d'ensemble de l'état
des routes de leur client, optimisant par la même occasion leur parcours.

\section{Analyse du marché}
\subsection{Demande}
La demande pour une telle solution est déjà présente, notamment en Valais.
La commune d'Ayent a déjà fait savoir son intérêt pour ce type de détection.

\subsection{Offres présentes sur le marché}
Actuellement, aucune offre comparable directement avec la nôtre n'est disponible.

\section{Analyse des partenaires}
\subsection{Eurocircuit}
Eurocircuit, leader européen dans la production de circuits imprimés sur mesures,
partenaire de choix pour la réalisation de produits électronique.
À noter que ce partenaire propose aussi un service de montage électronique,
ce qui nous permet de faire sous-traiter cette partie compliquée pour
un coût plus faible qu'en Suisse.

\subsection{Mädler}
Mädler est un leader dans l'industrie mécanique et propose des solutions adaptées à notre projet.
Cette entreprise sera notre partenaire pour les questions mécaniques du projet.

\subsection{Boschung}
%\begin{figure}[H]
%    \includegraphics[width=0.3\textwidth]{Images/business/boschung.jpg}
%\end{figure}
Boschung est une entreprise spécialisée dans les solutions de surveillance des routes et
notamment du déneigement. Ils proposent déjà un système de détection de verglas
sur les routes, mais sont relativement chers. Nous pourrions collaborer
afin de proposer nos solutions pour améliorer leurs gammes de produits.

\section{Analyse de la concurrence}
\subsection{Boschung}
Boschung, bien qu'un potentiel allié stratégique, pourrait devenir notre concurrent
principal. La détection de neige sur route est un domaine dans lequel ils cherchent à
développer. De plus ils possèdent un carnet de client bien fourni et un savoir-faire
très développé.

\subsection{Population de la région}
Certaines régions, notamment la région de Vex/Veysonnaz, jouissent d'un
excellent réseau d'alerte citoyenne. Plusieurs résidents sont des lève-tôt et
alertent donc automatiquement les services communaux ou privés de déneigement.
De telles régions n'ont donc aucun intérêt à utiliser notre solution, car
une solution est déjà présente et fait ses preuves chaque année.
