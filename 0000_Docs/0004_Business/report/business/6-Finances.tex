\section{Planification des coûts}
Les coûts liés au projet sont les suivants :
\begin{itemize}
    \item Trois ingénieurs à 110[CHF/heure]
    \item Un secrétaire à 60[CHF/heure]
    \item Un technicien à 80[CHF/heure]
    \item Un coût unitaire de fabrication à 250[CHF] :
    \begin{itemize}
        \item Le circuit imprimé à 9[CHF]
        \item Les composants pour un total de 140[CHF]
        \item Un boîtier mécanique en plastique moulé à 100[CHF]
    \end{itemize}
    \item Des coûts de Marketing liés aux déplacements, à la publicité ciblée
    \item Des coûts liés au site web ainsi qu'aux serveurs d'acquisition
\end{itemize}

\section{Bilan prévisionnel}
Un bilan prévisionnel sur 3 ans a été réalisé par trimestre.
Dans le pire des cas où le projet serait un échec total,
une perte de 750'000[CHF] est prévue.
Dans un bon cas, en vendant 100 unités à la fin de la 3e année,
et que le développement est complet, les coûts sont amortis. Et une
rentabilité sur 5 ans est envisageable.
Le bilan prévisionnel détaillé est disponible en annexe.

\section{Plan de liquidité}
Un somme de base de 1'000'000[CHF] est nécessaire pour développer
entièrement la solution. Cette somme peut provenir d'un partenaire
éventuel ou d'un investisseur.