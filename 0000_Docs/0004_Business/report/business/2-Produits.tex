\section{Description du produit}
LoRaSnow est un projet pilote de détection de hauteur de neige sur route, de débit
de chute de neige ainsi que des risque de verglas.
L'objectif est de fournir une solution de montioring de l'état d'un segment de route.
Implémenter un réseau de LoRaSnow permettrai donc de monitorer une région entière et de permettre
une meilleure gestion des ressources lors de la période hivernale. Tant bien pour une administration
publique que pour une entreprise privée.\\
LoRaSnow utilise le protocole LoRaWAN pour communiquer les données sur le cloud, permettant ainsi
d'être installé n'importe où sans nécessiter d'infrastructure (internet ou électricité) au préalable.\\
Lorsque la neige commence à tomber, ou qu'une couche importante de neige est présente
sur une route, une alerte est transmise au client, lui permettant d'être informer de la situation
sans devoir être actif devant un terminal. Cela permettrai d'éliminer les besoins de personnel
de piquet durant la nuit.


\section{Détection de hauteur de neige}
En utilisant un mélange de solutions lasers et de vision par ordinateur,
LoRaSnow permet une détection innovante et efficace de la couche de neige
présente sur un segment de route.

\section{Mesure du débit de chute de neige}
Toujours en utilisant la vision par ordinateur, une indication correcte et fiable
du débit de chute de neige en temps réel permet de se rendre compte des chutes de neige
sur la région et de ne pas se laisser surprendre.

\section{Détection du givre}
En intégrant une mesure de l'humidité, ainsi qu'une mesure de la température
du bitume, coupler aux prévisions météos, LoRaSnow offre une prévision
de verglas efficace.

\section{Pourquoi nous ?}
Nous apportons une solution simple et autonome à un problème complexe et inexploré.
Problème sur lequel nos prédécesseurs n'ont pas trouvé de solution fonctionnelle.