\section{Marketing}
\subsection{Tarifs}
Les tarifs pour notre produit sont les suivants :
\begin{itemize}
    \item Un prix unitaire de 6400[CHF]
    \item Des frais d'installation unitaire à 300[CHF]
    \item Un service de maintenance unitaire annuel de 500[CHF]
\end{itemize}

\subsection{Canaux}
\subsubsection{Publicité du produit}
Afin de faire connaître notre produit, nous allons nous baser sur trois
méthodes :
\begin{itemize}
    \item Bouche à oreille -> Proposer la solution à une commune
    et faire parler de notre réussite.
    \item Marketing ciblé -> Se présenter à des salons industriels
    et technologiques, montrer des exemples fonctionnels et parler de
    clients utilisant notre projet.
    \item Site web -> Présenter des vidéos de démonstrations,
    proposer un site en libre accès avec des données de capteurs en direct.
\end{itemize}

\section{Infrastructure}
\subsection{Serveurs et LoRaWAN}
Des serveurs d'acquisition de données seront mis en place,
et des Gateway LoRaWAN seront installées dans les régions ou nos
clients se situent.
Un serveur d'acquisition interne ainsi qu'une Gateway LoRaWAN
peuvent être implémentés chez le client directement.

\subsection{Site web}
Un site web devra être maintenu et mis à jour pour toujours fournir
des informations à jour aux potentiels clients.

\subsection{Service client}
Un service de dépannage et maintenance constant sera mis en place pour
les clients. Chaque automne une maintenance de routine sera réalisée
pour préparer le matériel à l'hiver. Si des capteurs tombent en panne
durant le fonctionnement, un service de dépannage sera disponible.

\section{Ressources humaines}
\subsection{R\&D}
Trois ingénieurs seront présent jusqu'à la fin de la première année
d'implémentation du projet pour développer et optimiser LoRaSnow.

\subsection{Service client}
Un technicien est nécessaire pour la fabrication,
l'installation et la maintenance des appareils.\\
Un secrétaire est prévu pour gérer les comptes de l'entreprise
ainsi que le service client.
