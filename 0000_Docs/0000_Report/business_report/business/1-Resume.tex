\section{Idée commerciale}
LoRaSnow apporte une solution efficace pour dégager efficacement les routes et permettre
aux automobilistes de conduire en toute sécurité, même par mauvais temps.\\
Il arrive parfois que les routes mettent du temps à être déneigées,
ou le salage est trop faible, conduisant à une chaussée glissante et dangereuse.
De plus, une répartition inhomogène du manteau neigeux sur une région peut rendre
la tâche compliquée, surtout en montagne. LoRaSnow apporte un monitoring constant
des niveaux de neige sur la route et du débit de neige à des points clés ainsi que
les possibilités de verglas, et ce, de manière efficace et rentable.\\[0.2cm]
Grâce à un réseau de capteurs sur une région,
il devient possible d'optimiser la courses des chasses-neige et de cibler les axes
en plus grandes difficultés, de même que d'offrir l'opportunité d'effectuer
des salages préventifs, avant que du verglas ne se forme.\\[0.2cm]
En plus de simplifier le déneigement des routes publiques, les entreprises de déneigement
peuvent bénéficier de ces avantages pour optimiser leur course chez leurs clients privés,
et par la même occasion faire des économies. Grâce à des capteurs intelligents
et une durée de vie maximale, LoRaSnow sera en mesure de vous fournir un débit
de chute de neige, un indicateur d'état de la route, une potentielle hauteur de
neige sur la chaussée ainsi qu'un indice de verglas. Le tout est relié sur le réseau
LoRaWAN, ce qui permet de consulter les données récoltées depuis n'importe où,
n'importe quand en ayant un simple accès au Web.

\section{Domaine d'activité}
Le domaine d'activité principal est la détection de neige sur route en montagne,
sur des routes peu accessibles.

\section{Marché}
Des communes, comme Ayent, ont manifesté leur intérêt pour une solution
de détection du niveau de neige sur route.
Des entreprises privées de déneigement bénéficieraient de LoRaSnow pour
optimiser leur service de déneigement et améliorer la qualité du service
fourni.\newpage

\section{Analyse des risques}
\subsection{Forces}
La force de LoRaSnow est qu'elle est l'unique solution disponible pour l'instant.
Des entreprises et la HEVs ont déjà travaillé sur le problème, sans succès.\\[0.2cm]
La solution est très peu coûteuse, et fonctionne en dehors de tout réseau électrique,
et n'as pas besoin de connexion internet grâce à la technologie LoRa.\\[0.2cm]
L'équipe regroupe toutes les aptitudes pour terminer entièrement le projet.
Trois étudiants passionnés, deux spécialisés en développement de systèmes embarqués
et un qualifié dans la mécanique.
L'équipe est accompagnée par des personnes hautement qualifiées capables de jugés le
projet de manière objective.

\subsection{Faiblesses}
L'équipe étant composée uniquement d'étudiants, elle n'a pas de notoriété dans le milieu.
Le système actuel n'est pas capable de mesurer des niveaux de neige en pleine journée,
ce qui n'est de toute manière pas nécessaire pour l'application principale.

\subsection{Opportunités}
LoRaSnow est la première solution fonctionnelle de détection de niveau de neige
sur route. Les communes du Valais sont déjà intéressées par cette solution.

\subsection{Menaces}
Si des personnes ou une entreprise avec plus d’influence que nous trouvaient
une solution aussi pour résoudre la problématique, l’équipe devrait trouver un 
moyen de se démarquer.
