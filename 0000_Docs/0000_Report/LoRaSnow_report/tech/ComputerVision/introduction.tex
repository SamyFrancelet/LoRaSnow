La \emph{Computer Vision} (ou vision par ordinateur) comprend l'acquisition, l'analyse et
le traitement des images numériques pour comprendre et extraire des données, informations
ou décisions.\\
Les applications et possibilités de ce domaine sont pour ainsi dire, infinies.

\section{Défis liés à la mesure en extérieur}
La principale difficulté de la mesure de neige par \emph{Computer Vision} est liée au fait
qu'il neige. Les caméras embarquées sont souvent de mauvaises qualités, et les débits
sont mauvais à cause des processeurs embarqués qui sont limités en puissance de calcul.

\begin{figure}[H]
    \centering
    \includegraphics[width=0.4\textwidth]{Images/computer_vision/perfect_snow.jpg}
    \includegraphics[width=0.4\textwidth]{Images/computer_vision/real_snow.png}
    \caption[Comparaison image de neige HD et cas concret]{Comparaison entre une image idéale \footnotemark[1] et un cas concret\footnotemark[2]}
    \label{fig:Snow comparison}
\end{figure}
\footnotetext[1]{Wallpaper from WallpaperCave, by caveman, \url{https://wallpapercave.com/w/scDoVwf} (last accessed: 20 January 2022)}
\footnotetext[2]{VibroSnow camera, installed at route du Pralan, Ayent, Suisse}

\noindent
Il faut donc trouver une méthode ne demandant pas trop de ressources de calcul, et pouvant fournir
des résultats utiles pour informer sur l'état des routes.
\subsection{Visibilité dans le brouillard et la nuit}
Bien qu'il est rare d'avoir une forte chute de neige ainsi qu'un brouillard épais, il reste nécessaire
de trouver un moyen d'améliorer la visibilité au travers du brouillard.
Selon ce rapport\cite{HAL}, les infrarouges proches \emph{(NIR)} permettent de voir au travers
d'un brouillard fin, et d'augementer \emph{légèrement} la visibilité au travers d'un brouillard épais.
L'ajout d'un éclairage infrarouge sera donc nécessaire, et permettra également de pouvoir augmenter
la visibilité durant la nuit.
\newpage

\section{Applications pour la mesure de neige}
Les possibilités de la \emph{Computer Vision} étant vastes, plusieurs méthodes ont été discutées.

\subsection{Mesure de niveau sur une règlette}
Les mesures de niveau de neige manuelles se font déjà avec une règlette plantée dans la neige.

\begin{figure}[H]
    \centering
    \includegraphics[width=0.3\textwidth]{Images/computer_vision/snow_meter.PNG}
    \caption[Mesure de neige à la règle]{Mesure d'environ 4.5 pouces de neige à Manitoba, Canada\footnotemark[1]}
    \label{fig:Snow meter}
\end{figure}
\footnotetext[1]{Image by Jerry Zachedniak, \url{https://ici.radio-canada.ca/nouvelle/1338531/hiver-tempete-parc-mont-riding}
(last accessed: 20 January 2022)}
\noindent
La mesure pourrait être réalisée en plaçant la caméra en face de la règlette.
Deux méthodes sont possibles :
\begin{description}
    \item[Avoir une règlette graduée] \hfill \\
    et compter le nombre de graduation encore visible pour déterminer la hauteur de neige.
    \item[Avoir un piquet d'une taille connue] \hfill \\
    comparer la hauteur de ce piquet au nombre de pixel quand il n'y a pas de neige,
    puis mesurer le nombre de pixels non-ensevelis pour mesurer la hauteur de neige.
\end{description}
\noindent
Cependant mesurer une règlette peut paraître simple d'un point de vue de l'implémentation,
mais présente plusieurs désavantages lors du fonctionnement:

\begin{description}
    \item[Il faut déneiger devant la caméra et la règlette] \hfill \\
    demandant probablement à un ouvrier de descendre de son chasse-neige pour déneiger l'installation.
    \item[L'installation ne doit pas être trop proche d'une route] \hfill \\ 
    de risque d'être ensevelie ou endommagée lors du passage d'un chasse-neige.
    \item[Si on utilise une règlette graduée] \hfill \\ 
    il faut s'assurer d'avoir un matériau surlequel la neige ne colle pas ou 
    ne réfléchis pas trop la lumière du soleil.
    \item[L'utilisation d'un piquet peut demander l'usage d'une intelligence artificielle \cite{SnowTimeLapse}] \hfill \\
    pour reconnaitre le piquet d'autres objets (p. ex.: arbres, lampadaires, ...)
    Cela demanderait trop de puissance de calcul pour un système embarqué basse consommation.
    Une autre solution serait de calibrer chaque installation pour reconnaître le piquet.
    Cependant la caméra étant embarquée, cette calibration serait certainement
    fastidieuse et demanderait une interface utilisateur supplémentaire pour la réaliser.
\end{description}
%\footnotetext[2]{\emph{Snow Depth Measurement via Time Lapse Photography and Automated Image Recognition},
%by Kevin S. J. Brown and Steven R. Fassnacht at Colorado State University,
%Departement of Ecosystem Science and Sustainability, 2019, http://www.codos.org/\#lit}
\newpage

\subsection{Mesure de niveau par téréovision}
La stéréovision est une méthode de mesure se servant d'images provenant de plusieurs points de vue.
Typiquement, deux caméras côte à côte, peuvent mesurer des profondeurs de la même manière que des yeux.

\begin{figure}[H]
    \centering
    \includegraphics[width=0.5\textwidth]{Images/computer_vision/stereovision.jpg}
    \caption[Schéma de principe stéréovision]{Schéma de principe de mesure de distance par stéréovision}
    \label{fig:Stereovision}
\end{figure}
\noindent
Beaucoup de caméras spécialisées dans la reconnaissance d'image par intelligence artificielle utilisent ce principe
pour estimer les distances (ex: OpenCV OAK cameras).
Cette méthode demande trop de puissance de calcul pour un système embarqué basse consommation et ne fournit
pas de résultats suffisamment précis pour mesurer une couche de neige.
\newpage

\subsection{Mesure du débit de chute de neige}
Une mesure simple et rapide consiste à estimer le débit de chute de neige en isolant les flocons.
\begin{figure}[H]
    \centering
    \includegraphics[width=0.35\textwidth]{Images/computer_vision/snow_cam.png}
    \includegraphics[width=0.35\textwidth]{Images/computer_vision/snowfall.png}
    \caption[Comparaison image source et bruit sur image]{Image source de la caméra VibroSnow\footnotemark[1] et image avec les flocons isolés}
    \label{fig:Snowfall}
\end{figure}
\footnotetext[1]{VibroSnow camera, installed at route du Pralan, Ayent, Suisse}
\noindent
Cette mesure, en parrallèle à une mesure de hauteur de neige, permettrait d'estimer 
l'augmentation de cette hauteur au fil du temps, fournissant ainsi une information supplémentaire
aux services de déneigement.

\subsection{Détection de route enneigée}
Étant donné que la mesure de hauteur de neige par \emph{Computer Vision} serait trop complexe
pour un système embarqué basse consommation, détecter si la route est enneigée ou non
permettrait de fournir une redondance à une mesure de hauteur de neige fournie par un autre capteur.
Si la petite zone mesurée par le capteur se retrouve mal déneigée, ou qu'un objet, comme une pierre, mal placé
fausse la mesure du capteur, savoir si le segment de route est enneigé ou non offre l'occasion d'éliminer ces erreurs.

\subsection{Méthodes retenues}
Les méthodes retenues pour ce projet de recherche sont \textbf{la mesure du débit de chute de neige}
et \textbf{la détection de route enneigée}. Elles peuvent fournir des informations pertinentes
pour le déneigement, tout en demandant relativement peu de puissance de calcul.
\newpage