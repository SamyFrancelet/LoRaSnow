\section{LiDAR}
Une batterie de tests a été effectuée afin d'attester du bon fonctionnement du LiDAR dans les conditions
attendues.\par
En effet, on remarque que le capteur est capable de fonctionner dans la plage utile de 2 à 
4 mètres. Il serait cependant préférable d'éviter d'avoir une distance trop grande entre ce dernier et 
la route, comme sa mesure diverge très vite de la réalité. \\
De la même manière, le LiDAR a été mis à l'épreuve dans le banc de test en laboratoire. Sur une série 
d'une centaine de mesures, il était parfaitement capable de mesurer une distance au sol, et ce peu importe 
le débit de fausse neige devant le capteur. Cette information est essentielle pour un fonctionnement en
condition réelle. De cette manière, nous savons que la neige ou le brouillard ne poseront pas de problèmes 
pour effectuer les mesures. La méthode de mesure du maximum a par ailleurs été retenue.\\
Comme ce projet est à même de devoir résister à de grandes variations de température entre le jour et
la nuit. Cependant, cette différence ne doit en aucun cas avoir des répectutions sur les mesures effectuées.
Heureusement, il semble que le capteur soit stable en température, restant majoritairement dans son 
erreur typique.\\
Après que les tests ci-dessus soient validés, nous avons pu nous concentrer sur une méthode de mesure de 
hauteur en laboratoire. Nous avons ainsi constaté que le capteur arrive à calculer la hauteur d'un élément,
même avec du bruit généré devant lui. Il faut cependant faire attention à la porosité du matériau mesuré. En
effet, certaines épaisseurs ont été fortement faussées. Cette étape a aussi permis de définitivement valider
la méthode du maximum pour le test suivant.\\
Finalement, le 4 décembre 2021, nous avons pu faire une nuit de mesure sur le terrain. Le LiDAR a été en
mesure de calculer la hauteur de neige présente sur la route, et ce avec une
précision de 7 milimètres! Ce test final a permis de valider le \emph{proof of concept} de cette méthode. De plus amples
recherches et essais doivent être effectués afin de totalement valider ou abandonner ce capteur, mais les
résultats actuels sont prometteurs.\par 
Il est important de noter que, comme prévu, le LiDAR ne fonctionne pas de jour, même lorsque le ciel
est complètement couvert. Cela ne pose en aucun cas un problème. En effet, ce projet est fait pour 
majoritairement fonctionner la nuit.

\subsection{Problèmes rencontrés}
Nous savons que le LiDAR a des difficultés à mesurer des distances sur des matériaux poreux. Cependant,
les essais sur le terrain avec de la véritable neige n'ont monté aucune erreur. Il serait nécessaire
d'approfondir les recherches afin de déterminer si certaines neiges peuvent provoquer ce problème.

\section{Computer Vision}
La \emph{Computer Vision} qui devait initialement servir plutôt de redondance au \emph{LiDAR} fournit
finalement des résultats très suprenant avec des algorithmes très simples.\\
La mesure du débit de chute de neige est fonctionnelle, bien que perfectible avec des données météos précises.\\
La détection de route enneigée est une réussite, l'algorithme est très simple et peut tourner sur un
processeur rudimentaire. Elle fournit une mesure très importante et le fait à la perfection.\\
La vision par ordinateur permet donc de palier parfaitement aux problèmes du \emph{LiDAR},
c'est à dire : les mesures durant la journée, et les erreurs liées à la mesure sur un point (p. ex: caillou sur le point de mesure).\\
Si on imaginait un système encore plus simpliste, la \emph{Computer Vision} pourrait être le seul outil de mesure.

\subsection{Problèmes rencontrés}
\subsubsection{Caméra}
Au départ, pour tester rapidement l'idée de la mesure de débit, nous avions imaginés filmer notre session
de mesure avec un ordinateur, puis comparer les données de bruit du LiDAR avec le débit estimé par
la caméra. Hors les webcams utilisées n'avaient pas de possibilités de désactivation de l'autofocus,
du réglage automatique du diaphragme et rendaient les vidéos capturées inutilisables.\\
Il a donc été primordiale de récupérer des données d'une caméra bien plus fiable, et sur le terrain.
Nous remercions encore une fois Dr. Mudry Pierre-André et M. Matter Fabien de nous avoir donné
l'accès aux vidéos de la caméra du projet parent \emph{VibroSnow\cite{VibroSnow}}.
\subsubsection{Données météos}
Une demande d'accès aux données de \emph{MeteoSwiss} a été visée et signée par M. Cyrille Bezençon et transmise
par la poste vers \emph{MeteoSwiss}. Cependant l'autorisation a tardé à arriver et les données météos
n'ont pas pu être utilisées pour améliorer la reconnaissance du débit de chute de neige.

\section{Mécanique}
Grace à un banc de test comprenant une cage d’essai et un canon à confettis. Plusieurs séries de mesures 
concluantes ont pu être réalisées. Le débit de neige produit par le canon étant réglable, il augmente les 
possibilités de mesures. La cage permet de contenir les nuages de confettis sans dégrader les conditions 
de travails des autres usagers de la salle.\\
Différents designs de boitier ont été imaginés avant d’arriver à la solution finale. L’étanchéité fut la 
partie plus problématique lors de la conception. L’utilisation de joint torique pincé permit de résoudre 
ces problèmes. L’ABS étant une matière très simple à injecter, la fabrication des pièces du boitier serait 
corolaire. Le support du boitier permet de régler à la fois l’élévation et l’azimut du boitier. La partie 
amovible supérieure permet un très bon accès à la partie du module électronique. Elle se verrouille facilement 
grâce à une fermeture par grenouillère. L’ouverture est verrouillée grâce à la serrure incrustée dans la 
fixation.\\
Le module électronique se fixe simplement dans le boitier à l’aide de 4 vis. Une pièce mécanique permet 
de loger les batteries. Au-dessus des batteries, une plaque permet la fixation du capteur Lidar 
et des PCB. Suffisamment d’espace est prévu autour du module afin de pouvoir disposer aisément le câblage. 
La simplicité du module permet une manipulation rapide et simple des différents éléments.