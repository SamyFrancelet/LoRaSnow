\section{Mécanique}
Grace à un banc de test comprenant une cage d’essai et un canon à confettis. Plusieurs séries de mesures 
concluantes ont pu être réalisées. Le débit de neige produit par le canon étant réglable, il augmente les 
possibilités de mesures. La cage permet de contenir les nuages de confettis sans dégrader les conditions 
de travails des autres usagers de la salle.\\
Différents designs de boitier ont été imaginés avant d’arriver à la solution finale. L’étanchéité fut la 
partie plus problématique lors de la conception. L’utilisation de joint torique pincé permit de résoudre 
ces problèmes. L’ABS étant une matière très simple à injecter, la fabrication des pièces du boitier serait 
corolaire. Le support du boitier permet de régler à la fois l’élévation et l’azimut du boitier. La partie 
amovible supérieure permet un très bon accès à la partie du module électronique. Elle se verrouille facilement 
grâce à une fermeture par grenouillère. L’ouverture est verrouillée grâce à la serrure incrustée dans la 
fixation.\\
Le module électronique se fixe simplement dans le boitier à l’aide de 4 vis. Une pièce mécanique permet 
de loger les batteries. Au-dessus des batteries, une plaque permet la fixation du capteur Lidar 
et des PCB. Suffisamment d’espace est prévu autour du module afin de pouvoir disposer aisément le câblage. 
La simplicité du module permet une manipulation rapide et simple des différents éléments.