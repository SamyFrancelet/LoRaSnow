La \emph{Computer Vision} (ou vision par ordinateur) comprend l'acquisition, l'analyse et
le traitement des images numériques pour comprendre et extraire des données, informations
ou décisions.\\
Les applications et possibilités de ce domaine sont pour ainsi dire, infinie.

\section{Défis liés à la mesure en extérieur}
La principale difficulté de la mesure de neige par \emph{Computer Vision} est lié au fait
qu'il neige. Les caméras embarquées sont souvent de mauvaise qualité, et les débits
sont mauvais à cause des processeurs embarqués qui sont limités en puissance de calcul.

\begin{figure}[H]
    \centering
    \includegraphics[width=0.4\textwidth]{Images/computer_vision/perfect_snow.jpg}
    \includegraphics[width=0.4\textwidth]{Images/computer_vision/real_snow.png}
    \caption[]{Comparaison entre une image idéale \footnotemark[1] et un cas concret\footnotemark[2]}
    \label{Snow comparison}
\end{figure}
\footnotetext[1]{Wallpaper from WallpaperCave, by caveman, https://wallpapercave.com/w/scDoVwf
(last accessed: 20 January 2022)}
\footnotetext[2]{VibroSnow camera, implemented at route du Pralan, Ayent, Suisse}

\noindent
Il faut donc trouver une méthode ne demandant pas trop de ressources de calcul, et pouvant fournir
des résultats utiles pour informer sur l'état des routes.

\section{Applications pour la mesure de neige}
Les possibilités de la \emph{Computer Vision} étant vaste, plusieurs méthodes ont été discutées.

\newpage

\subsection{Mesure de niveau sur une règlette}
Les mesures de niveau de neige manuel se font déjà avec une réglette plantée dans la neige.

\begin{figure}[H]
    \centering
    \includegraphics[width=0.3\textwidth]{Images/computer_vision/snow_meter.PNG}
    \caption[]{Mesure d'environ 4.5 pouces de neige à Manitoba, Canada\footnotemark[1]}
    \label{Snow meter}
\end{figure}
\footnotetext[1]{Image by Jerry Zachedniak, https://ici.radio-canada.ca/nouvelle/1338531/hiver-tempete-parc-mont-riding
(last accessed: 20 January 2022)}
\noindent
La mesure pourrait être réalisé en plaçant la caméra en face de la règlette.
Deux méthodes sont possibles :
\begin{description}
    \item[Avoir une règlette graduée] \hfill \\
    et compter le nombre de graduation encore visible pour déterminer la hauteur de neige.
    \item[Avoir un piquet d'une taille connue] \hfill \\
    comparer la hauteur de ce piquet au nombre de pixel quand il n'y a pas de neige,
    puis mesurer le nombre de pixels non-ensevelis pour mesurer la hauteur de neige.
\end{description}
\noindent
Cependant mesurer une règlette peut paraître simple d'un point de vue de l'implémentation,
mais présente plusieurs désavantages lors du fonctionnement:

\begin{description}
    \item[Il faut déneiger devant la caméra et la règlette] \hfill \\
    demandant probablement à un ouvrier de descendre de son chasse-neige pour déneiger l'installation.
    \item[L'installation ne doit pas être trop proche d'une route] \hfill \\ 
    de risque d'être ensevelie ou endommagée lors du passage d'un chasse-neige.
    \item[Si on utilise une règlette graduée] \hfill \\ 
    il faut s'assurer d'avoir un matériau surlequel la neige ne colle pas ou 
    ne réfléchis pas trop la lumière du soleil.
    \item[L'utilisation d'un piquet peut demander l'usage d'une intelligence artificielle] \hfill \\
    pour reconnaitre le piquet d'autres objets (p. ex.: arbres, lampadaires, ...) (ref\footnotemark[2])
    Cela demanderait trop de puissance de calcul pour un système embarqué basse consommation.
    Une autre solution serait de calibrer chaque installation pour reconnaître le piquet.
    Cependant la caméra étant intégrée dans un système embarqué, cette calibration serait certainement
    fastidieuse et demanderait une interface utilisateur supplémentaire pour la réaliser.
\end{description}
\footnotetext[2]{\emph{Snow Depth Measurement via Time Lapse Photography and Automated Image Recognition},
by Kevin S. J. Brown and Steven R. Fassnacht at Colorado State University,
Departement of Ecosystem Science and Sustainability, 2019, http://www.codos.org/\#lit}
\newpage

\subsection{Mesure de niveau par Stéréovision}
La Stéréovision est une méthode de mesure se servant d'images provenant de plusieurs point de vue.
Typiquement, deux caméras côte à côte, peuvent mesurer des profondeurs comme des yeux.

\begin{figure}[H]
    \centering
    \includegraphics[width=0.5\textwidth]{Images/computer_vision/stereovision.jpg}
    \caption[]{Schéma de principe de mesure de distance par stéréovision}
    \label{Stereovision}
\end{figure}
\noindent
Beaucoup de caméra spécialisée dans la reconnaissance d'image par Intelligence artificielle utilise se principe
pour estimer les distances (ex: OpenCV OAK cameras).
Cette méthode demande trop de puissance de calcul pour un système embarqué basse consommation et ne fournit
pas de résultat suffisamment précis pour mesurer une couche de neige.
\newpage

\subsection{Mesure du débit de chute de neige}
Une mesure simple et rapide est d'estimer le débit de chute de neige en isolant les flocons.
\begin{figure}[H]
    \centering
    \includegraphics[width=0.35\textwidth]{Images/computer_vision/snow_cam.png}
    \includegraphics[width=0.35\textwidth]{Images/computer_vision/snowfall.png}
    \caption[]{Image source de la caméra VibroSnow\footnotemark[1] et image avec les flocons isolés}
    \label{Snowfall}
\end{figure}
\footnotetext[1]{VibroSnow camera, implemented at route du Pralan, Ayent, Suisse}
\noindent
Cette mesure, en parrallèle à une mesure de hauteur de neige, permettrait d'estimer 
l'augmentation de cette hauteur au fil du temps. Fournissant ainsi une information supplémentaire
aux services de déneigement.

\subsection{Détection de route enneigée}
Étant donné que la mesure de hauteur de neige par \emph{Computer Vision} serait trop complexe
pour un système embarqué basse consommation, détecter si la route est enneigée ou non
permettrait de fournir une redondance à une mesure de hauteur de neige fournie par un autre capteur.
Si la petite zone mesurée par le capteur se retrouve mal déneigée, ou qu'un objet, comme un caillou, mal placé
fausse la mesure du capteur, savoir si le segment de route est enneigé ou non permettrait d'éliminer ces erreurs.

\subsection{Méthodes retenues}
Les méthodes retenues pour ce projet de recherche sont \textbf{la mesure du débit de chute de neige}
et \textbf{la détection de route enneigée}. Elles peuvent fournir des informations pertinentes
pour le déneigement, tout en demandant relativement peu de puissance de calcul.
\newpage

\section{Implémentation}
Pour tester les algorithmes de mesure avec des vidéos sur le terrain,
Dr. Mudry Pierre-André et M. Matter Fabien nous on aimablement laissé accès à
la caméra de notre projet parent \emph{VibroSnow\cite{VibroSnow}} et nous
les remercions énormément.

\subsection{Récupération des vidéos}
La caméra de \emph{VibroSnow} détecte le passage d'objet (voitures, chute de neige,...) et
enregistre une vidéo qui est ensuite transmise à un serveur \emph{Windows}.
Bien que l'accès aux vidéos nous a été donné, nous ne pouvons pas aller chercher les vidéos
directement sur le serveur \emph{Windows} car il est utilisé pour d'autres projets auxquels nous
n'avons pas accès.\\
Il a donc fallu créer un script \emph{Powershell}, transferant chaque jour les vidéos
cumulées sur le serveur \emph{Windows} vers un serveur auquel nous avons accès.
Un \emph{Raspberry Pi} a été mis en place comme serveur pour récuperer les vidéos.

\subsection{Mesure du débit de chute de neige}
La méthode utilisée pour détecter les chutes de neige se décompose ainsi :
\begin{description}
    \item[Soustraction de deux images] \hfill \\
    pour isoler les éléments qui ont bougé entre les deux images
    \item[Seuillage des niveaux de blancs sur l'image] \hfill \\
    pour accentuer les chutes de neige
    \item[Calcul du ratio de pixels blancs] \hfill \\
    pour avoir un nombre correspondant au débit de chute de neige    
\end{description}

\begin{figure}[H]
    \begin{subfigure}{.45\textwidth}
        \includegraphics[width=\linewidth]{Images/computer_vision/snowfall/original.png}
        \caption{Image originale}
        \label{fig:Snowfall_original}
    \end{subfigure}
    \hfill
    \begin{subfigure}{.45\textwidth}
        \includegraphics[width=\linewidth]{Images/computer_vision/snowfall/noise.png}
        \caption{Image avec neige isolée}
        \label{fig:Snowfall_noise}
    \end{subfigure}
    \hfill
    \centering
    \begin{subfigure}{.45\textwidth}
        \includegraphics[width=\linewidth]{Images/computer_vision/snowfall/snowfall.png}
        \caption{Image seuillée avec calcul du ratio de pixels blancs (9.05\% ici)}
        \label{fig:Snowfall_thres}
    \end{subfigure}
    \caption{Étapes de la mesure de débit de chute de neige}
    \label{fig:Snowfall_algorithm}
\end{figure}
\newpage

\subsection{Détection de route enneigée}
Deux méthodes ont été testées pour détecter si la route est enneigée ou non.\\
La première réalise un simple seuillage des niveaux de blancs, et un calcul
du ratio des pixels blancs sur l'image. On récupère plusieurs images et on
calcul la moyenne du ratio de blanc sur toute les images.
Cette moyenne est ensuite comparée à une moyenne similaire réalisée sur une
vidéo de la route déneigée, en vérifiant qu'on se trouve au même moment de
la journée (jour/nuit, matin/après-midi).\\
La deuxième est identique, à l'exception d'une suppression du bruit réalisée
avant le seuillage. Cette suppresion du bruit reprends la méthode d'isolation
de neige utilisée pour mesurer le débit de chute de neige et soustrait cette image
de bruit à l'image originale. Cette méthode demande un peu plus de calculs mais peut
potentiellement générer un résultat plus fiable.

\begin{figure}[H]
    \begin{subfigure}{.45\textwidth}
        \includegraphics[width=\linewidth]{Images/computer_vision/snowOnRoad/ref_original.png}
        \caption{Image de référence originale}
        \label{fig:SnowOnRoad_ref_original}
    \end{subfigure}
    \hfill
    \begin{subfigure}{.45\textwidth}
        \includegraphics[width=\linewidth]{Images/computer_vision/snowOnRoad/ref_thres.png}
        \caption{Image de référence seuillée}
        \label{fig:SnowOnRoad_ref_thres}
    \end{subfigure}
    \caption{Image de référence}
    \label{fig:SnowOnRoad_ref}
\end{figure}

\begin{figure}[H]
    \begin{subfigure}{.45\textwidth}
        \includegraphics[width=\linewidth]{Images/computer_vision/snowOnRoad/test_original.png}
        \caption{Image de test originale}
        \label{fig:SnowOnRoad_test_original}
    \end{subfigure}
    \hfill
    \begin{subfigure}{.45\textwidth}
        \includegraphics[width=\linewidth]{Images/computer_vision/snowOnRoad/test_thres.png}
        \caption{Image de test seuillée}
        \label{fig:SnowOnRoad_test_thres}
    \end{subfigure}
    \hfill
    \begin{subfigure}{.45\textwidth}
        \includegraphics[width=\linewidth]{Images/computer_vision/snowOnRoad/test_denoised.png}
        \caption{Image de test avec suppression de bruit}
        \label{fig:SnowOnRoad_test_denoised}
    \end{subfigure}
    \hfill
    \begin{subfigure}{.45\textwidth}
        \includegraphics[width=\linewidth]{Images/computer_vision/snowOnRoad/test_denoisedThres.png}
        \caption{Image de test avec suppression de bruit et seuillée}
        \label{fig:SnowOnRoad_test_denoisedThres}
    \end{subfigure}
    \caption{Image de test}
    \label{fig:SnowOnRoad_test}
\end{figure}
\newpage


\section{Résultats}