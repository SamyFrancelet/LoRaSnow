\begin{summary}
\section{Contexte}

Ne vous est-il jamais arrivé de vous réveiller un matin et, lorsque vous prenez votre véhicule ou un 
transport en commun, surprise ! Il a neigé durant la nuit, et les services de déneigement n'y étaient
visiblement pas préparés.\\
LoRaSnow est un projet qui vise à palier à ce problème. Intégrant les dernières technologies de l'internet 
des objets et une utilisation intelligente de capteurs, ces modules alertent les services privés et
communaux de chutes de neige dès les premiers flocons, permettant une réaction rapide et efficace.\\
En effet, le système d'alerte de piquet en place ne permet pas une vue globale de la région, et tarde
parfois à réagir. De plus, les prévisions météorologiques sont parfois imprécises quand il s'agit de 
quantité de précipitation. Ce projet amène donc une surveillance constante et automatique d'une région,
remplaçant par la même occasion le système coûteux déjà en place.

\end{summary}