\section{Résultats}
Plusieurs scripts \emph{Python} ont été réalisés pour testés les différentes
méthodes de mesure. Cette section détails ces scripts, les tests effectués
et des conclusions préliminaires.

\subsection{Mesure de débit de chute de neige}
Cette mesure doit permettre une information estimée de la chute de neige
actuelle. Idéalement, on devrait pouvoir estimer les précipitations de neige
en mm/h. Mais n'ayant pas encore reçu l'accès aux données de \emph{MeteoSwiss}
pour comparer nos mesures à des données fiables, une approche simplifiée a été
appliquée.\\
On récupère plusieurs vidéos qu'on trie selon 3 critères : \label{criterias}
\begin{description}
    \item[Le moment de la journée] \hfill \\
    matin, journée, soir, nuit
    \item[La quantité visuelle de neiger qui est en train de tomber] \hfill \\
    rien, petite neige, neige, grosse neige, tempête
    \item[L'état de la route] \hfill \\
    sèche, déneigée, partiellement enneigée, complêtement enneigée
\end{description}
Chaque vidéo avec une route sèche ou déneigée, et avec aucune chute de neige
peut servir de référence pour faire les mesures suivantes.
Il est important de comparer des vidéos se passant durant la même période de la journée.
La luminosité peut grandement affecter le résultat de la mesure.

\subsection{Détection de route enneigée}
Le but de cette mesure est de fournir, a minima, si oui ou non la route
est enneigée. Idéalement elle devrait pouvoir différencier plusieurs degré
d'enneigement de la route. Ces degrés sont définis selon les critères
cités précédemment (\ref{criterias})