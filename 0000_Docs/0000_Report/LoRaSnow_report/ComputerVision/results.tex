\section{Résultats}
Plusieurs scripts \emph{Python} ont été réalisés pour testés les différentes
méthodes de mesure. Cette section détails ces scripts, les tests effectués
et des conclusions préliminaires.

\subsection{Mesure de débit de chute de neige} \label{snowfall}
Cette mesure doit permettre une information estimée de la chute de neige
actuelle. Idéalement, on devrait pouvoir estimer les précipitations de neige
en mm/h. Mais n'ayant pas encore reçu l'accès aux données de \emph{MeteoSwiss}
pour comparer nos mesures à des données fiables, une approche simplifiée a été
appliquée.\\
On récupère plusieurs vidéos qu'on trie selon 2 critères :
\begin{description}
    \item[Le moment de la journée] \hfill \\
    journée, nuit
    \item[La quantité visuelle de neige qui est en train de tomber] \hfill \\
    rien, petite neige, neige, grosse neige
\end{description}
Il est important de comparer des vidéos se passant durant la même période de la journée.
La luminosité peut grandement affecter le résultat de la mesure. La localisation
de la caméra (Ayent) ne permet pas de différencier significativement la matinée
de la journée et la soirée de la nuit. Il a donc été convenu de séparer uniquement
par jour/nuit.

\subsection{Détection de route enneigée}
Le but de cette mesure est de fournir, a minima, si oui ou non la route
est enneigée. Idéalement elle devrait pouvoir différencier plusieurs degré
d'enneigement de la route. \\
Pour réaliser les mesures on récupère plusieurs vidéos qu'on trie ainsi :
\begin{description}
    \item[Le moment de la journée] \hfill \\
    journée, nuit 
    \item[L'état de la route] \hfill \\
    déneigée, partiellement enneigée, complêtement enneigée 
\end{description}
Chaque vidéo sans chute de neige avec une route déneigée peut servir de 
référence pour différencier une route déneigée d'une route enneigée.
Dans l'idéal, on devrait aussi pouvoir les trier selon la quantité de chute
de neige, mais n'ayant pas suffisamment de vidéos pour réaliser ce tri, on devra
se contenter de cette méthode.
Les deux méthode de mesure citées au point \ref{snowOnRoad} vont être mises
à l'épreuve pour déterminer laquelle est la plus pertinente pour ce projet.