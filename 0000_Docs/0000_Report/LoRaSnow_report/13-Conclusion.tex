\section{Bilan}

Beaucoup d'étapes ont été réalisées afin d'atteindre le but fixé pour ce projet pilote, c'est-à-dire 
mesurer une hauteur de neige dans des conditions difficiles. Au fur et à mesure de l'avancement de LoRaSnow,
nous nous sommes rendu compte de la vraie signification de \emph{conditions difficiles}. En effet, faire 
fonctionner des systèmes dans un froid glacial, du vent et de la neige est un défi qui demande des
ressources et de l'imagination.\\
Nous sommes finalement arrivés avec deux solutions complémentaires; l'une mesurant directement une hauteur 
de neige et l'autre mesurant des débits ainsi que l'état de la route. Cette redondance apporte une fiabilité
supplémentaire. De plus, une conception mécanique astucieuse permet une protection efficace des éléments 
électroniques.\\
Une bonne gestion du planning a été réalisée. En effet, malgré un léger retard dans la partie
recherche, nous avons réussi à gagner une certaine avance dans les autres tâches, ce qui nous a permis de 
se concentrer sur des détails pour notamment peaufiner les tests en extérieur. On retrouve les plannings prévus et 
réalisés à l'annexe \ref{app:planning}.\par 
Il reste cependant encore une longue route avant la commercialisation d'un tel produit. En effet, ce projet 
pilote doit servir de base pour de futurs améliorations, visant on l'espère, à un projet abouti. Une analyse
préalable du marché et des opportunités a d'ailleurs été effectuée. Cela nous a permis de voir au-delà du
simple projet technique, en nous projetant dans le monde de la gestion de projet.\par 
Finalement, nous nous sommes beaucoup investis dans ce projet, même hors du cadre du projet pilote. De 
nombreuses heures ont été consacrées à trouver des solutions innovantes et intelligentes. Nous pouvons sans 
aucun doute dire que nous sommes ressortis enrichis de cette expérience.

\section{Signatures}

\vspace{\fill}
Vincent Savioz \hfill Samy Francelet \hfill Gabriel Deferr
\vspace{\fill}