\section{Discussion des résultats}

Une batterie de tests a été effectuée afin d'attester du bon fonctionnement du LiDAR dans les conditions
attendues.\par
En effet, on remarque que le capteur est capable de fonctionner dans la plage utile de 2 à 
4 mètres. Il serait cependant préférable d'éviter d'avoir une distance trop grande entre ce dernier et 
la route, comme sa mesure diverge très vite de la réalité. \\
De la même manière, le LiDAR a été mis à l'épreuve dans le banc de test en laboratoire. Sur une série 
d'une centaine de mesures, il était parfaitement capable de mesurer une distance au sol, et ce peu importe 
le débit de fausse neige devant le capteur. Cette information est essentielle pour un fonctionnement en
condition réelle. De cette manière, nous savons que la neige ou le brouillard ne poseront pas de problèmes 
pour effectuer les mesures. La méthode de mesure du maximum a par ailleurs été retenue.\\
Comme ce projet est à même de devoir résister à de grandes variations de température entre le jour et
la nuit. Cependant, cette différence ne doit en aucun cas avoir des répectutions sur les mesures effectuées.
Heureusement, il semble que le capteur soit stable en température, restant majoritairement dans son 
erreur typique.\\
Après que les tests ci-dessus soient validés, nous avons pu nous concentrer sur une méthode de mesure de 
hauteur en laboratoire. Nous avons ainsi constaté que le capteur arrive à calculer la hauteur d'un élément,
même avec du bruit généré devant lui. Il faut cependant faire attention à la porosité du matériau mesuré. En
effet, certaines épaisseurs ont été fortement faussées. Cette étape a aussi permis de définitivement valider
la méthode du maximum pour le test suivant.\\
Finalement, le 4 décembre 2021, nous avons pu faire une nuit de mesure sur le terrain. Le LiDAR a été en
mesure de calculer la hauteur de neige présente sur la route, et ce avec une
précision de 7 milimètres! Ce test final a permis de valider le \emph{proof of concept} de cette méthode. De plus amples
recherches et essais doivent être effectués afin de totalement valider ou abandonner ce capteur, mais les
résultats actuels sont prometteurs.\par 
Il est important de noter que, comme prévu, le LiDAR ne fonctionne pas de jour, même lorsque le ciel
est complètement couvert. Cela ne pose en aucun cas un problème. En effet, ce projet est fait pour 
majoritairement fonctionner la nuit.

\section{Problèmes rencontrés}

\begin{description}
    \item[Porosité du matériau mesuré] \hfill \\ 
    Nous savons que le LiDAR a des difficultés à mesurer des distances sur des matériaux poreux. Cependant,
    les essais sur le terrain avec de la véritable neige n'ont monté aucune erreur. Il serait nécessaire
    d'approfondir les recherches afin de déterminer si certaines neiges peuvent provoquer ce problème.
\end{description}