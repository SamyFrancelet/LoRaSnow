\section{Discussion des résultats}

Une batterie de tests a été effectuée afin d'attester du bon fonctionnement du LiDAR dans les conditions
attendues.\par
En effet, on remarque que le capteur est capable de fonctionner dans la plage utile de 2 à 
4 mètres. Il serait cependant préférable d'éviter d'avoir une distance trop grande entre ce dernier et 
la route, comme sa mesure diverge très vite de la réalité. \\
De la même manière, le LiDAR a été mis à l'épreuve dans le banc de test en laboratoire. Sur une série 
d'une centaine de mesures, il était parfaitement capable de mesurer une distance au sol, et ce peu importe 
le débit de fausse neige devant le capteur. Cette information est essentielle pour un fonctionnement en
condition réelle. De cette manière, nous savons que la neige ou le brouillard ne poseront pas de problèmes 
pour effectuer les mesures. La méthode de mesure du maximum a par ailleurs été retenue.\\


\section{Problèmes rencontrés}